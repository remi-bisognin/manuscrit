%!TeX encoding = UTF-8
% Metadata for coverpages
\thesisname{Thèse de doctorat}
\gradename{docteur}
\univ{de l’université Pierre et Marie Curie}
\logos{./ma_couverture/Logo_DepPhys-ENS_Blanc}{./ma_couverture/Logo_officiel_de_Sorbonne_Universite}{./ma_couverture/CNRS}                   % Un à trois logo. Le 1er est celui de \univ
\specialite{Physique}
\ecoledoctnum{564}
\ecoledoct{Physique en Île-de-France}
\title{Measuring elementary quantum excitations in 1D conductor}
\titleen{...}

% si nécessaire, pour les métadonnées
%\titlemeta{La laine des Dupondt au Pays de l'or noir} 
%\titlemetaen{Dupondt's wool in "Land of Black Gold" } 

\date{1\up{er} Juillet 2019}
\author{Rémi BISOGNIN}
\advisor{Gwendal Fève}
\atinstitution{à l’École normale supérieure}
\atlab{{\Large au Laboratoire de Physique de l'ENS}}
% Jury as LaTeX Tabular. Pas de président avant soutenance								  
\jury{ %                         
M./M\up{me} & quelqu'un & Rapporteur        \\
M./M\up{me} & quelqu'un & Rapporteur        \\
M./M\up{me} & quelqu'un  & Examinateur           \\
M\up{me}& quelqu'un & Examinatrice \\
M./M\up{me} & Gwendal Fève & Directeur de thèse }

\makeatletter
\@ifpackageloaded{thcover-psl}{\logos{PSL}{tintin}{}}{\relax}
\makeatother

\resume{le résumé en français \par\hfill  (1700 car max, espaces inclus)}
\motscles{les mots clefs en français }

% si nécessaire, pour les métadonnées
%\titlemetaen{Dupondt's wool in "Land of Black Gold" } 

\abstract{ le résumé en anglais
\par\hfill (1700 chars max, spaces included)}
\keywords{les mots clefs en anglais }
